% !TEX root = thesis.tex

Recent decades have seen a tremendous rise in the affordability and
performance of various computational technologies, enabling
researchers to propose and probe ever more complicated numerical
models. These simulations often generate incredible quantities of data
that must be sifted through to glean useful conclusions. This thesis
highlights our efforts to automate this process in two specific areas:
(a) uncovering simplified descriptions of dynamic network models and
(b) detecting important parameter combinations in general nonlinear
systems. Both advances involve modification of the manifold learning algorithm, Diffusion Maps (DMAPS), to
address the particular problem.

In the first case, the challenge is quanitifying the similarity of two
networks in a reasonable amount of computational time. We propose a
number of possible solutions, and examine their performance when
combined with DMAPS. We find that by combining suitable measures of
similarity with DMAPS we are able to uncover low-dimensional structure
in a set of networks, thus enabling us to describe the system in terms
of one or two values instead of thousands.

In the second, we must extend Diffusion Maps to operate on the graph
of a function. In particular we consider a model that maps parameter
values to some output. By properly formulating the DMAPS kernel, we enable DMAPS to discover the directions
in parameter space along which model predictions vary most
significantly.

Both result in algorithms that we hope are practically useful to
researchers in a variety of fields who are looking for simplified
descriptions of their complex systems.
