% !TEX root = ../../thesis.tex

\section{Equation-free modeling\label{sec:ef}}

The second key component in the work that follows is Equation Free
(EF) modeling. This framework enables one to analyze agent-based
simulations such as social network evolution or molecular dynamics
using systems-level algorithms without deriving explicit, analytical
equations governing the properties of interest. EF modeling has been
used to accelerate simulations, to locate coarse steady states, and to
perform bifurcation analysis, using only the sort of fine-scale
simulations that researchers across disciplines increasingly rely on
to model their problems~\ref{all sorts of EF and fine scale stuff}. We
will now provide an overview of the general framework, and a particular
application known as coarse projective integration.

EF techniques rely on two complementary algorithmic elements: a
restriction operator $\Res$ that maps fine-scale simulations to coarse
variables, and a lifting operator $\Li$ that maps coarse variables to
consistent fine-scale simulations. Thus lifting is the inverse of
restriction, so $\Res \cdot \Li (x) = x$. We will denote the state of
the fine system by $u(t)$, with $U(t)$ the corresponding coarse
state. $u(t)$ could represent the evolving state of neurons in a
simulation of brain activity, while $U(t)$ may track the average
action potential. In such complex models it is difficult to derive an
explicit formula governing the progression of $U(t)$, but by
periodically measuring this quantity as the simulation progresses
through $\Res: u \rightarrow U$, we can estimate its trajectory. Then,
using an integration scheme as simple as forward Euler, we may
approximate the value of $U(t)$ at some future time,
$U(t + \Delta t)$. Using $\Li: U \rightarrow u$ we obtain a
fine-scaled state $u(t + \Delta t)$. This process of alternately
restricting and lifting can be repeated until the system has reached a
desired state. Overall, the result is an accelerated simulation as we
have replaced $\Delta t$ expensive, full-system steps with a single,
cheap application of forward Euler during each iteration. This method
is known as coarse projective integration (CPI), and it is illustrated
in Fig.~\ref{fig:cpi-ill}.

\begin{figure}
  \centering
  \includegraphics[width=0.8\textwidth]{cpi-intro}
  \caption[Illustration of coarse projective integration]{Illustration
    of CPI procedure in which repeated applications of the restriction
    operator $\Res$ are used to approximate the trajectory of $U(t)$
    and to estimate its value at some future time $U(t + \Delta
    t)$. Subsequent application of $\Li$ allows one to iterate this
    process, accelerating simulations. \label{fig:cpi-ill}}
\end{figure}
