\section{Outline of thesis}

We begin with a background of the mathematical techniques used
throughout the bulk of the thesis work, including manifold learning
algorithms and the Equation Free framework. With this foundation
established, we turn to an extension of these techniques into the
field of network science. In particular, we study a dynamic multigraph
model which exhibits the sort of preferential attachment phenomenon
seen in many real-world social networks. We implement a coarse
projective integrator for this system and use it to construct a
stationary state solver. We also explore the problem of quantifying
similarities between pairs of networks, and having found such a
measure we embed whole trajectories of network evolution and find
low-dimensional structure. 

After validating the use of manifold learning in this simpler network
setting, we apply similar techniques to a popular model of disease
propogation, the susceptible-infected-susceptible (SIS) model. Again,
by selecting an appropriate measure of similarity between networks we
are able to algorithmically reduce the problem's dimensionality using
DMAPS without recourse to analytical derivations. Our results suggest
that a previously-derived low dimensional system may be improved.

The third and final section of this thesis moves from network models
to more traditional systems of ordinary differential equations whose
evolution is governed by certain sets of parameters. Here, the goal is
to determine which parameters have the largest influence on model
predictions so that we may ignore those parameter combinations that do
not significantly affect the model output, thus reducing the
dimensionality of parameter space. By constructing a variant of DMAPS
that operates on model predictions, we are able to realize this
objective even in the presence of nonlinear parameter combinations. In
a related section, we also show how such an approach can identify the
fast manifolds in singularly perturbed systems.

In total, these chapters display the utility of extending manifold
learning techniques into novel domains, and of the pervasiveness of
low-dimensional structure in complex
systems.

%%% Local Variables: ***
%%% mode:latex ***
%%% TeX-master: "../../thesis.tex"  ***
%%% End: ***